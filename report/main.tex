\documentclass[11pt,a4paper]{article}

\usepackage[utf8]{inputenc}
\usepackage[T1]{fontenc}
\usepackage{amsmath, amssymb}
\usepackage{geometry}
\usepackage{hyperref}

\geometry{margin=2.5cm}

\title{Kryptografia stosowana: Szyfrowanie RSA i bezpieczna komunikacja}
\author{Michał Zduńczyk \and Daniel Wybranowski \and Mateusz Hutorowicz}
\date{}

\begin{document}
\maketitle

% =========================
\section{Wstęp}
% (a)
% =========================

Wprowadzenie — Przegląd RSA i cele kryptograficzne projektu

Celem projektu było zaprojektowanie oraz implementacja kryptosystemu RSA, który jest jednym z najważniejszych i najczęściej stosowanych algorytmów kryptografii asymetrycznej. Algorytm RSA umożliwia bezpieczne przesyłanie informacji przez niezabezpieczony kanał komunikacyjny bez konieczności wcześniejszego współdzielenia tajnego klucza między stronami.

Głównym problemem rozwiązywanym w projekcie było bezpieczne przesłanie wiadomości przez niezaufane środowisko komunikacyjne przy użyciu zaimplementowanego systemu szyfrowania. Rozwiązanie polega na wygenerowaniu pary kluczy — publicznego i prywatnego — gdzie klucz publiczny służy do szyfrowania danych, a klucz prywatny do ich odszyfrowania.

W ramach projektu zaimplementowano mechanizm generowania kluczy, proces szyfrowania oraz odszyfrowania wiadomości. Implementacja pokazuje, jak matematyczne podstawy kryptografii mogą zostać przełożone na działający system informatyczny. Projekt ma również na celu zrozumienie różnicy między kryptografią symetryczną i asymetryczną oraz praktycznych aspektów bezpieczeństwa danych.

% =========================


\section{Podstawy matematyczne}
% (b)
% =========================

Podstawy matematyczne — Fundamenty działania RSA

Algorytm RSA opiera się na kilku kluczowych pojęciach matematycznych: liczbach pierwszych, arytmetyce modularnej oraz funkcjach jednokierunkowych.

Podstawą działania RSA jest wybór dwóch dużych liczb pierwszych. Liczby te są następnie mnożone, tworząc moduł 
n, który stanowi część klucza publicznego i prywatnego. Mnożenie dużych liczb pierwszych jest operacją obliczeniowo łatwą, natomiast rozkład ich iloczynu na czynniki pierwsze jest bardzo trudny dla odpowiednio dużych wartości. Bezpieczeństwo RSA wynika właśnie z trudności problemu faktoryzacji dużych liczb.

Drugim istotnym elementem jest arytmetyka modularna, czyli wykonywanie działań z resztą z dzielenia. Zarówno szyfrowanie, jak i odszyfrowanie w RSA polega na potęgowaniu modulo n. Do efektywnego wykonywania takich obliczeń stosuje się algorytmy szybkiego potęgowania modularnego, które pozwalają pracować na bardzo dużych liczbach.

W procesie generowania kluczy wykorzystuje się również funkcję Eulera 𝜑(𝑛), która określa liczbę liczb względnie pierwszych z 
n. Wybierany jest wykładnik publiczny e, który jest względnie pierwszy z 𝜑(𝑛), a następnie obliczany jest wykładnik prywatny 
𝑑 jako odwrotność modularna liczby 𝑒 modulo 𝜑(𝑛).Dzięki temu operacje szyfrowania i odszyfrowania są wzajemnie odwracalne.

RSA jest przykładem funkcji jednokierunkowej z tzw. „furtką” (trapdoor). Oznacza to, że łatwo jest wykonać operację szyfrowania przy użyciu klucza publicznego, natomiast odwrócenie jej bez znajomości klucza prywatnego jest obliczeniowo niepraktyczne.

RSA należy do kryptografii asymetrycznej — używa dwóch różnych kluczy. W przeciwieństwie do niej kryptografia symetryczna (np. AES) wykorzystuje jeden wspólny klucz do szyfrowania i odszyfrowania. Szyfry symetryczne są znacznie szybsze, ale wymagają bezpiecznego sposobu przekazania klucza. W praktyce często stosuje się systemy hybrydowe: RSA do wymiany klucza, a AES do szyfrowania danych.

Należy podkreślić, że bezpieczeństwo RSA opiera się fundamentalnie na trudności faktoryzacji dużych liczb — jest to podstawowe założenie bezpieczeństwa tego algorytmu.

% =========================


\section{Projekt i implementacja}
% (c)
% =========================

W ramach projektu zaimplementowano system bezpiecznej komunikacji oparty na kryptosystemie klucza publicznego RSA. System został zaprojektowany w sposób modularny, oddzielając warstwę matematyczno--kryptograficzną od warstwy demonstracyjnej i testowej. Implementacja obejmuje generowanie kluczy RSA, szyfrowanie i deszyfrowanie danych, testowanie poprawności algorytmów oraz prosty protokół komunikacji symulujący bezpieczną wymianę wiadomości.

Generowanie kluczy RSA polega na losowaniu dwóch dużych liczb pierwszych o długości 1024 bitów. Do sprawdzania ich pierwszości zastosowano probabilistyczny test Millera--Rabina, który zapewnia wysokie prawdopodobieństwo poprawności przy akceptowalnym czasie obliczeń. Następnie obliczany jest moduł
\( n = p \cdot q \) oraz funkcja Eulera
\( \varphi(n) = (p - 1)(q - 1) \).
Wykładnik publiczny \( e \) wybierany jest jako standardowa wartość 65537, natomiast wykładnik prywatny \( d \) wyznaczany jest przy użyciu rozszerzonego algorytmu Euklidesa poprzez obliczenie odwrotności modularnej.

Szyfrowanie i deszyfrowanie danych realizowane są zgodnie z definicją algorytmu RSA. Do obliczeń potęg modularnych wykorzystano algorytm szybkiego potęgowania (exponentiation by squaring), co znacząco zwiększa wydajność obliczeń dla dużych wykładników. Operacje arytmetyczne na dużych liczbach całkowitych realizowane są przy użyciu biblioteki \texttt{\path{boost::multiprecision::cpp\_int}}.

W celu obsługi wiadomości tekstowych zastosowano uproszczony mechanizm kodowania polegający na zamianie znaków ASCII na wartości liczbowe. Każdy znak szyfrowany jest osobno, co pozwala spełnić warunek \( m < n \) bez stosowania złożonych schematów dopełniania (paddingu).

Demonstracyjny protokół bezpiecznej komunikacji został zaimplementowany w pliku
\texttt{\path{secure_demo.cpp}}. Symuluje on wymianę wiadomości pomiędzy nadawcą i odbiorcą poprzez generowanie kluczy, publikację klucza publicznego, szyfrowanie wiadomości, zapis szyfrogramu do pliku oraz jego odszyfrowanie po stronie odbiorcy.

\subsection*{Pseudocode: RSA Key Generation}
\begin{verbatim}
Algorithm RSA-Key-Generation
Input: key length
Output: public key (e, n), private key (d, n)

1. Generate two large random primes p and q
2. Compute n = p * q
3. Compute phi(n) = (p - 1)(q - 1)
4. Choose e such that gcd(e, phi(n)) = 1
5. Compute d = e^(-1) mod phi(n)
6. Return (e, n) and (d, n)
\end{verbatim}

\subsection*{Pseudocode: RSA Encryption}
\begin{verbatim}
Algorithm RSA-Encrypt
Input: message m, public key (e, n)
Output: ciphertext c

1. Ensure m < n
2. Compute c = m^e mod n
3. Return c
\end{verbatim}

\subsection*{Pseudocode: RSA Decryption}
\begin{verbatim}
Algorithm RSA-Decrypt
Input: ciphertext c, private key (d, n)
Output: message m

1. Compute m = c^d mod n
2. Return m
\end{verbatim}

\subsection*{Pseudocode: Secure Communication Protocol}
\begin{verbatim}
Algorithm Secure-Communication
1. Receiver generates RSA key pair
2. Receiver publishes public key (e, n)
3. Sender encrypts message using public key
4. Ciphertext is transmitted over public channel
5. Receiver decrypts message using private key
\end{verbatim}

% =========================
\section{Wyniki}
% (d)
% =========================

Poprawność implementacji została zweryfikowana przy użyciu testów matematycznych oraz testów funkcjonalnych. Rozszerzony algorytm Euklidesa został przetestowany na dużej liczbie losowych par liczb, co potwierdziło poprawność obliczeń największego wspólnego dzielnika oraz spełnienie tożsamości Bézouta.

Działanie kryptosystemu RSA sprawdzono poprzez szyfrowanie i deszyfrowanie wiadomości tekstowych. Odszyfrowana wiadomość była identyczna z wiadomością oryginalną, co potwierdza poprawność implementacji algorytmu RSA. Najbardziej czasochłonnym etapem działania systemu jest generowanie kluczy, natomiast szyfrowanie i deszyfrowanie krótkich wiadomości przebiega wystarczająco szybko dla celów demonstracyjnych. Ograniczeniem rozwiązania jest brak obsługi dużych ilości danych.

\subsection*{Przykład działania systemu}

Poprawność działania systemu została również zweryfikowana na podstawie
rzeczywistego uruchomienia programu demonstracyjnego
(\texttt{secure\_demo.cpp}). Poniżej przedstawiono przykładowy wynik
uzyskany podczas testu systemu.

\begin{itemize}
\item \textbf{Wiadomość jawna (plaintext):} HELLO
\item \textbf{Wiadomość po odszyfrowaniu (decrypted plaintext):} HELLO
\end{itemize}

Podczas szyfrowania każdy znak wiadomości został zaszyfrowany osobno
przy użyciu algorytmu RSA. Wygenerowany szyfrogram stanowił ciąg dużych
liczb całkowitych, zapisanych do pliku \texttt{cipher.txt}. Fragment
rzeczywistego szyfrogramu przedstawiono poniżej:

\begin{verbatim}
14432898923990352452009045344983720966622271983973591131000170864005536...
10343580874497184732684271618570625033912501502479511875915177614092409...
20038361361298919327328562897599194252421215161097741995486639385839009...
\end{verbatim}

Odszyfrowana wiadomość jest identyczna z wiadomością oryginalną, co
jednoznacznie potwierdza poprawność implementacji algorytmu RSA oraz
zastosowanego protokołu bezpiecznej komunikacji.

% =========================
\section{Analiza bezpieczeństwa}
% (e)
% =========================

Bezpieczeństwo kryptosystemu RSA opiera się na trudności faktoryzacji dużych liczb złożonych. Przy zastosowanej długości klucza 1024 bity system spełnia cele demonstracyjne projektu, jednak w rzeczywistych zastosowaniach zalecane są obecnie dłuższe klucze.

Implementacja ma charakter edukacyjny i zawiera uproszczenia. Kodowanie wiadomości odbywa się bez użycia formalnych schematów dopełniania, takich jak PKCS\#1, które w realnych systemach kryptograficznych są niezbędne do ochrony przed atakami opartymi na strukturze danych. Dodatkowo klucze kryptograficzne zapisywane są do plików tekstowych, co w praktycznych zastosowaniach stanowiłoby istotne zagrożenie bezpieczeństwa.

W rzeczywistych systemach RSA wykorzystywane jest zazwyczaj wyłącznie do bezpiecznej wymiany kluczy, natomiast dane szyfrowane są przy użyciu algorytmów symetrycznych, takich jak AES. Takie podejście, znane jako szyfrowanie hybrydowe, znacząco poprawia wydajność i bezpieczeństwo systemu.

% =========================
\section{Podsumowanie}

% (f)
% =========================
Podsumowanie — Wnioski i możliwe rozszerzenia

Projekt pozwolił na praktyczne poznanie działania kryptosystemu RSA oraz zrozumienie matematycznych podstaw kryptografii asymetrycznej. Podczas realizacji zaimplementowano generowanie kluczy, szyfrowanie i odszyfrowanie wiadomości, co umożliwiło przełożenie teorii na działający system.

W trakcie pracy wykorzystano pojęcia takie jak liczby pierwsze, arytmetyka modularna, odwrotność modularna oraz szybkie potęgowanie. Projekt pokazał również, jak istotne są parametry bezpieczeństwa, w szczególności długość klucza oraz jakość generowanych liczb losowych.

Możliwe przyszłe rozszerzenia projektu obejmują zastosowanie większych długości kluczy, co zwiększyłoby poziom bezpieczeństwa systemu. Kolejnym krokiem mogłaby być implementacja podpisów cyfrowych z użyciem RSA, co pozwoliłoby zapewnić nie tylko poufność, ale także autentyczność i integralność danych.

System można również rozbudować do postaci kryptosystemu hybrydowego, w którym RSA służy do bezpiecznej wymiany klucza symetrycznego, a właściwe dane szyfrowane są szybkim algorytmem symetrycznym, takim jak AES. Dodatkowo można dodać mechanizmy paddingu (np. OAEP), bezpieczne generatory liczb losowych oraz optymalizacje wydajnościowe.

Takie rozszerzenia przybliżyłyby projekt do rozwiązań stosowanych w rzeczywistych systemach kryptograficznych.
% =========================

\end{document}
